\documentclass[11pt]{article} 

% packages with special commands
\usepackage{amssymb, amsmath}
\usepackage{epsfig}
\usepackage{array}
\usepackage{ifthen}
\usepackage{color}
\usepackage{fancyhdr}
\usepackage{graphicx}
\usepackage{mathtools}
%\usepackage{algorithm}
%\usepackage{algpseudocode}
%\usepackage{mdframed}
%\newmdtheoremenv{lem}{Lemma}
\definecolor{grey}{rgb}{0.5,0.5,0.5}

\begin{document}
\newcommand{\tr}{\text{tr}}
\newcommand{\E}{\textbf{E}}
\newcommand{\diag}{\text{diag}}
\newcommand{\argmax}{\text{argmax}}
\newcommand{\Cov}{\text{Cov}}
\newcommand{\Var}{\text{Var}}
\renewcommand{\thefootnote}{\fnsymbol{footnote}}

\begin{center}
\noindent Charles Zheng EE 378b HW 3
\end{center}

\noindent\textbf{1.}
a.
Since the space of unit vectors is compact,
there exists $x^*$ such that $||M||_2 = ||Mx^*||_2$.
For the same reason, there exist unit vectors $x^\circ, y^\circ$ such that
\[
\langle x^\circ, M y^\circ \rangle = \max_{||x|| = ||y|| = 1} \langle x, My \rangle
\]
Letting $y^* = Mx^*/||M||_2$, we have $||y||_2 = 1$.
Therefore
\[
||M||_2 = y^* M x^* \leq \max_{||y||=||x||=1} \langle y, Mx \rangle
\]


\noindent\textbf{2.}

\noindent\textbf{3.}

\noindent\textbf{4.}

\noindent\textbf{5.}


\end{document}
