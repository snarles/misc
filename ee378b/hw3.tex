\documentclass[11pt]{article} 

% packages with special commands
\usepackage{amssymb, amsmath}
\usepackage{epsfig}
\usepackage{array}
\usepackage{ifthen}
\usepackage{color}
\usepackage{fancyhdr}
\usepackage{graphicx}
\usepackage{mathtools}
%\usepackage{algorithm}
%\usepackage{algpseudocode}
%\usepackage{mdframed}
%\newmdtheoremenv{lem}{Lemma}
\definecolor{grey}{rgb}{0.5,0.5,0.5}

\begin{document}
\newcommand{\tr}{\text{tr}}
\newcommand{\E}{\textbf{E}}
\newcommand{\diag}{\text{diag}}
\newcommand{\argmax}{\text{argmax}}
\newcommand{\Cov}{\text{Cov}}
\newcommand{\Var}{\text{Var}}
\renewcommand{\thefootnote}{\fnsymbol{footnote}}

\begin{center}
\noindent Charles Zheng EE 378b HW 3
\end{center}

\noindent\textbf{1.}
a.
Since the space of unit vectors is compact,
there exists $x^*$ such that $||M||_2 = ||Mx^*||_2$.
For the same reason, there exist unit vectors $x^\circ, y^\circ$ such that
\[
\langle x^\circ, M y^\circ \rangle = \max_{||x|| = ||y|| = 1} \langle x, My \rangle
\]

Letting $y^* = Mx^*/||M||_2$, we have $||y||_2 = 1$.
Therefore
\[
||M||_2 = y^* M x^* \leq \max_{||y||=||x||=1} \langle y, Mx \rangle
\]

Also, by Cauchy-Schwarz we have 
\begin{align*}
\max_{||x|| = ||y|| = 1} \langle x, My \rangle
= (x^\circ)^* M y^\circ
\leq ||x^\circ||_2 ||My^\circ||_2 \leq ||My^\circ||_2 
\leq \max_{||y||=1} ||My||_2 = ||M||_2
\end{align*}

Having shown that \[||M||_2 \leq \max_{||x|| = ||y|| = 1} \langle x, My \rangle \leq ||M||_2\]
we conclude that the definitions are equivalent

b.  Let the SVD of $M$ be written $M = UDV^T$ where $D =
diag(\sigma_1,\hdots, \sigma_n)$.
Let $v_1$ be the first column of $V$, then $||v_1||_2 = 1$ and
\[
||Mv_1||_2 = ||UDV^T v_1||_2 = ||UD e_1||_2 = ||De_1||_2 = \sigma_1
\]
Hence
\[
\sigma_1 = ||Mv_1||_2 \leq \max_{||x|| = 1} ||Mx||_2 = ||M||_2
\]

Meanwhile for any unit vector $x$, defining $y = V^T x$ we have $||y||_2 \leq 1$.
Then
\[
\max_{||x||=1} ||Mx||_2 = \max_{||x|| = 1} ||UDV^T x||_2
\leq \max_{||x|| = 1} ||DV^T x||_2 \leq \max_{||y|| = 1} ||Dy||_2
\]
But defining $a_i = y_i^2$,
\[
\max_{||y||=1} ||Dy||_2^2 = \max_{||y||=1} \sum_{i=1}^n \sigma_i^2 y_i^2
= \max_{\sum a_i = 1, a_i \geq 0} \sum_{i=1}^n \sigma_i^2 a_i
\]
is maximized by $a = e_1$, hence $\max_{||y|| = 1} ||Dy||_2 = \sigma_1$.

Having shown that
\[
\sigma_1 \leq ||M||_2 \leq \sigma_1
\]
we conclude that the two definitions are equivalent.\\

\noindent\textbf{2.}
a.
From 1a we have
\[
||M^*||_2 = \max_{||x||=||y||=1} \langle x, M^* y \rangle
= \max_{||x||=||y||=1} \langle y, M x \rangle = ||M||_2
\]

b.
We have
\[
||AB||_2 = \max_{||x|| = 1} ||ABx||_2 = \max_{y = Bx\text{ for some }||x|| = 1} ||Ay||_2
\]
Meanwhile, if $y = Bx$, and $||x||_2 = 1$, we have $||y||_2 \leq ||B||_2$.
Therefore the set
$\{y: y = Bx \text{ for some }x\text{ such that }||x||=1\}$
is contained in the set
$\{y: ||y||_2 \leq ||B||_2\}$.
Hence
\[
\max_{y = Bx\text{ for some }||x|| = 1} ||Ay||_2 \leq
\max_{||y|| = ||B||_2} ||Ay||_2 = ||A||_2 ||B||_2
\]

\noindent\textbf{3.}
i. 
\[
||aM||_2 = \max_{||x||=1} ||aMx||_2 = \max_{||x||=1} |a| ||M_2 x||_2 = |a|\max_{||x||=1} ||M x||_2 = a||M||_2
\]

ii.
\[
||A + B||_2 = \max_{||x|| = 1} ||Ax + Bx||_2 \leq \max_{||x||=1} ||Ax||_2 + ||Bx||_2
\leq \max_{||x|| = 1} ||Ax||_2 + \max_{||x||=1} ||Bx||_2 = ||A||_2 + ||B||_2
\]

iii.
Proof of contrapositive:
If $M \neq 0$, then some row $M_i$ is nonzero.
But then $||Me_i||_2 = ||M_i||_2 > 0$, so $||M||_2 > 0$.\\


\noindent\textbf{4.}

\noindent\textbf{5.}


\end{document}
