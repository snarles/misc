\documentclass[11pt]{article} 

% packages with special commands
\usepackage{amssymb, amsmath}
\usepackage{epsfig}
\usepackage{array}
\usepackage{ifthen}
\usepackage{color}
\usepackage{fancyhdr}
\usepackage{graphicx}
\usepackage{mathtools}
\definecolor{grey}{rgb}{0.5,0.5,0.5}

\begin{document}
\newcommand{\tr}{\text{tr}}
\newcommand{\E}{\textbf{E}}
\newcommand{\diag}{\text{diag}}
\newcommand{\argmax}{\text{argmax}}
\newcommand{\Cov}{\text{Cov}}
\pagestyle{fancy}
\fancyhead[L]{CHARLES ZHENG}

\section{Gaussian hypercontractivity}

In this report we review the derivation of Nelson's hypercontractive
inequality for the $L^2$ function space under gaussian measure.
Hypercontractive inequalities are important for deriving
various concentration of measure results, but we refer to the
interested reader to [LeDoux] and [Chatterjee] for the motivation
behind hypercontractivity.

We take material from [LeDoux] and [Chatterjee], but add details and
intuition when possible.

\subsection{Preliminaries}

Consider the Gaussian measure,
\[ \mu(A) = C_n \int_A e^{-U(x)} dt \]
for $U(x) = -||x||_2^2/2$, where $C_n$ is a normalizing constant
(explicitly, $C_n = (2\pi)^{-n/2}$).
Define the operator
\[
L = \Delta - \nabla U \cdot \nabla
\]
Define the Dirichlet form $\mathcal{E}(f,g) = \mu(f(-Lg))$.
We claim that
\begin{equation}\label{innerproductf}
\mathcal{E}(f,g) =\mu(f(-Lg))=\int f(-Lg) d\mu = \int \nabla f \cdot \nabla g d \mu
\end{equation}

We will now derive \eqref{innerproductf}. Recall that if $\Omega$ is an open
bounded subset of $\mathbb{R}^n$ with smooth boundary $\Gamma$,
$\hat{v}$ is the unit surface normal to $\Gamma$, $u$ a
function and ${\bf v}$ a vector-valued function, both continuously
differentiable,  then
\begin{equation}\label{intparts}
\int_\Omega \nabla u \cdot {\bf v}(x) d\Omega = \int_\Gamma u({\bf
  v}\cdot \hat{v}) d\Gamma-\int_\Omega
u \nabla \cdot {\bf v} d\Omega
\end{equation}

Note that 
\begin{align*}
\nabla \cdot ((\nabla g(x)) e^{-U(x)}) 
&= (\nabla \cdot \nabla g(x))e^{-U(x)} + \nabla g(x) \cdot (\nabla
e^{-U(x)}) 
\\&= \Delta g(x) e^{-U(x)} + \nabla g(x) \cdot (-\nabla U(x)
e^{-U(x)})
\\&= (\Delta g(x) - \nabla g(x) \cdot \nabla U(x)) e^{-U(x)}
\\&= Lg(x) e^{-U(x)}
\end{align*}
Furthermore, note that $\nabla g(x) e^{-U(x)}$ vanishes as $||x||_2
\to \infty$.
Therefore applying integration py parts, we can ignore the boundary
term in \eqref{intparts} and obtain
\begin{align*}
\int_{\mathbb{R}^n} f(x) (-Lg(x)) e^{-U(x)} dx
&= -\int_{\mathbb{R}^n} f(x) \nabla\cdot ((\nabla g(x)) e^{-U(x)}) dx
\\&= \int_{\mathbb{R}^n} \nabla f(x) \cdot ((\nabla g(x)) e^{-U(x)}) dx
\end{align*}
which, up to normalization terms, gives $\int f(-Lg)d\mu = \int \nabla
f \cdot \nabla g d\mu$.

$L$ is an infinitesimal generator for the semigroup of operators
$P_t$, by the \emph{heat} equation
\[
\frac{\partial}{\partial t} P_t f(x) |_{t=u} = L P_u f(x)
\]
which is also written as $\partial_t P_t = L P_t$.
Alternatively,
\[
P_t = e^{tL} = \sum_{k=0}^\infty \frac{t^k}{k!} L^k
\]
Here is an intuitive explanation for understanding the equivalence of
the two forms.
For small $\delta$, we know that by definition, $P_{t+\delta} f
\approx f + \delta L P_t f$.
However, to get an accurate approximation for $P_{t+u}$ for $u$ large,
we should first approximate 
\[P_{t + (u/k)} \approx f + \frac{u}{k} LP_t
f = (I + \frac{u}{k}) f\]
where $I$ is the identity operator, then approximate
\[P_{t + 2(u/k)} \approx f + \frac{u}{k} L P_{t+(u/k)} f =
(I+\frac{u}{k}) P_{t + (u/k)} f \approx (I + \frac{u}{k}L)^2 f\]
and so on,
hence
\[
P_{t+u} f \approx (I + \frac{u}{k}L)^k f
\]
Taking $k \to \infty$, we get the
exponential form $P_{t+u} = e^{uL}P_t$.
The semigroup property follows automatically from the exponential form:
\[P_{t+s} = e^{(t+s)L} = e^{tL} e^{sL} = P_t P_s\]
given the appropriate conditions for the convergence of the infinite
series, etc.

\subsection{The Ornstein-Uhlembeck Semigroup}

In the case of the gaussian measure, where $U(x) = -||x||^2/2$,
we know the explicit form of $P_t$:
\[
P_t f(x) = C_n \int f(e^{-t}x + \sqrt{1-e^{-2t}}z) e^{-||z||^2/} dz =
\textbf{E}[f(e^{-t}x + \sqrt{1-e^{-2t}}Z)]
\]

An important consequence is that
\begin{equation}\label{limitE}
\lim_{t \to \infty} f(x) = \textbf{E}[f(Z)] = \gamma^n(f).
\end{equation}

Another consequence is that for any $t \geq 0$,
\begin{equation}\label{intE}
\E[P_t f(Z)] = \textbf{E}[f(e^{-t}Z + \sqrt{1-e^{-2t}}Z')] = \textbf{E}[f(Z)] = \gamma^n(f)
\end{equation} 
where $Z, Z'$ are independent standard normal variates,
hence $e^{-t}Z + \sqrt{1-e^{-2t}}Z'$ has the same distribution as $Z$.

Let us verify that $P_t$ satisfies the heat equation.
On one hand, assuming we can differentiate under the integral sign,
\begin{align}
\frac{\partial}{\partial t} P_t f(x) &= \textbf{E}\left[
\frac{\partial}{\partial t} f(e^{-t}x + \sqrt{1-e^{-2t}}Z)\right]
\\&= \textbf{E}\left[
\nabla f(e^{-t}x + \sqrt{1-e^{-2t}}Z)
\left(e^{-t} + \frac{e^{-2t}}{\sqrt{1-e^{-2t}}}Z\right)
\right]
\\&= e^{-t}x\textbf{E}[\nabla f(e^{-t}x + \sqrt{1-e^{-2t}}Z)] +
\frac{e^{-2t}}{\sqrt{1-e^{-2t}}} \textbf{E}[Z \nabla f(e^{-t}x +
\sqrt{1-e^{-2t}}Z)]
\intertext{Using the identity $\textbf{E}Zg(Z) = \textbf{E}g'(Z)$}
&= e^{-t}x\textbf{E}[\nabla f(e^{-t}x + \sqrt{1-e^{-2t}}Z)] +
\frac{e^{-2t}}{\sqrt{1-e^{-2t}}} \textbf{E}\left[\sum_{i=1}^n
\frac{\partial^2}{\partial Z_i^2} f'(e^{-t}x +
\sqrt{1-e^{-2t}}Z)\right]
\\&= e^{-t}x\textbf{E}[\nabla f(e^{-t}x + \sqrt{1-e^{-2t}}Z)] +
\frac{e^{-2t}}{\sqrt{1-e^{-2t}}} \textbf{E}\left[\sum_{i=1}^n
\sqrt{1-e^{-2t}} \frac{\partial^2 f (w)}{\partial w_i^2}\bigg|_{w = e^{-t}x +
\sqrt{1-e^{-2t}}Z}\right]
\\&= e^{-t}x\textbf{E}[\nabla f(e^{-t}x + \sqrt{1-e^{-2t}}Z)] +
e^{-2t} \textbf{E}\left[\sum_{i=1}^n \frac{\partial^2 f (w)}{\partial w_i^2}\bigg|_{w = e^{-t}x +
\sqrt{1-e^{-2t}}Z}\right]
\\&= e^{-t}x\E[\nabla f(e^{-t}x + \sqrt{1-e^{-2t}}Z)] +
e^{-2t} \E\left[\Delta f(e^{-t}x +\sqrt{1-e^{-2t}}Z)\right]
\label{dtside}
\end{align}

On the other hand,
\begin{align}
LP_t f(x) &= \Delta P_t f(x) + x \cdot \nabla P_t f(x)
\\&= \Delta \E[f(e^{-t}x + \sqrt{1-e^{-2t}}Z)] + x \cdot
\nabla \E[f(e^{-t}x + \sqrt{1-e^{-2t}}Z)]
\\&= \E\left[\sum_{i=1}^n \frac{\partial^2}{\partial x_i^2}
  f(e^{-t}x + \sqrt{1-e^{-2t}}Z)\right] + \sum_{i=1}^n \E\left[x_i
\frac{\partial}{\partial x_i}f(e^{-t}x + \sqrt{1-e^{-2t}}Z)\right]
\\&= \E\left[\sum_{i=1}^n e^{-2t} f''(e^{-t}x +
  \sqrt{1-e^{-2t}}Z)\right] + \sum_{i=1}^n \E\left[e^{-t} x_i f'(e^{-t}x + \sqrt{1-e^{-2t}}Z)\right]
\\&= e^{-2t} \E\left[\Delta f(e^{-t}x
  +\sqrt{1-e^{-2t}}Z)\right] + 
e^{-t}x\E[\nabla f(e^{-t}x + \sqrt{1-e^{-2t}}Z)]
\end{align}
which matches \eqref{dtside}.

\subsection{Ornstein-Uhlembeck process}

The OU semigroup can also be written as
\[
P_t f(x) = \E[f(X_t)]
\]
where $X_t$ is the \emph{Ornstein-Uhlembeck process} with $X_0 = x$.

The OU process is defined by
\[
X_t = e^{-t} X_0 + e^{-t} W_{e^{2t}-1}
\]
where $W_t$ is standard Brownian motion.

An easy consequence of the representation $P_t f(x) =
\E[f(X_t)]$ is the following.
For $g$ positive,
\begin{equation}\label{csstep}
|P_t f(x)|^2 \leq (P_t g (x)) (P_t \frac{f^2}{g}(x))
\end{equation}
which follows from the Cauchy-Schwartz inequality,
\begin{align}
|P_t f(x)|^2 &= \E^2\left[\frac{f(X_t)}{\sqrt{g}(X_t)} \sqrt{g}(X_t)\right]
\\&\leq \E\left[ \frac{f^2(X_t)}{g(X_t)}\right] \E[g(X_t)] = (P_t \frac{f^2}{g}(x))(P_t g (x))
\end{align}

However, as we will not use any other property of the OU process in the proof of
hypercontractivity, we will refer the interested reader to Karatzas
and Shreve (1991) for more details.

\subsection{Proof of Hypercontractivity}

The hypercontractive inequality for the OU semigroup was first proved
by Nelson (1973): we now state his result.


\noindent\textbf{Proposition.}
\emph{Let $P_t$ be the OU semigroup. For any $p > 1$, and $t > 0$, there exists a
  $q = q(t,p) > p$ such that for any $f \in L^2(\mu)$, the following
  holds:}
\[
||P_t f||_{L^q(\mu)} \leq ||f||_{L^2(\mu)}
\]
\emph{Furthermore, the above holds with $q(t,p) = 1+(p-1)e^{2t}$.}

We follow the proof in Chatterjee (2014), which makes use of the
logarithmic Sobolev inequality for gaussian measures,

\noindent\textbf{Lemma.} (Gaussian log Sobolev inequality)
\emph{Let $\gamma^n$ be the standard gaussian measure in
  $\mathbb{R}^n$.
Then if $f: \mathbb{R}^n \to \mathbb{R}$ is an absolutely continuous
function, then}
\[
\gamma^n\left(f^2 \log \frac{f^2}{\gamma^n (f^2)}\right) \leq \gamma^n(2||\nabla f||_2^2)
\]
\emph{where $\gamma^n(\cdot)$ denotes expectation wrt the measure $\gamma^n$.}

Recall also the identity \eqref{innerproductf} which implies that
$\gamma^n((f)(Lg)) = -\gamma^n(\nabla f\cdot \nabla g)$.

\noindent\textbf{Proof of Lemma.}

Begin by defining $v = f^2$.
Then
\begin{align}
\gamma^n\left(f^2 \log \frac{f^2}{\gamma^n (f^2)}\right) 
&=\gamma^n\left(v \log \frac{v}{\gamma^n (v)}\right)
\\&= \gamma^n(v \log v) - \gamma^n(v)\log \gamma^n(v)
\\&= \gamma^n(v \log v) - \gamma^n(v \log \gamma^n(v))
\intertext{using the property \eqref{limitE}}
&= \gamma^n((P_0 v)(\log P_0 v)) - \gamma^n((P_\infty v)(\log
P_\infty v))
\\&= -\int_0^\infty \frac{\partial}{\partial t} \gamma^n((P_t v)(\log
P_t v)) dt
\\&= -\int_0^\infty  \gamma^n\left( ( \frac{\partial}{\partial t} P_t v)(\log
P_t v)\right) + \gamma^n\left( (  P_t v)( \frac{\partial}{\partial t}\log
 P_t v)\right) dt
\\&= -\int_0^\infty  \gamma^n\left( ( L P_t v)(\log
P_t v)\right) + \gamma^n\left( (  P_t v)( \frac{LP_t v}{
 P_t v})\right) dt
\\&= -\int_0^\infty \gamma^n((LP_t v) (1+\log P_t v)) dt
\\&= \int_0^\infty \mathcal{E}(P_t v, 1+\log P_t v) dt
\\&= \int_0^\infty \gamma^n((\nabla P_t v) \cdot (\nabla \log P_t v)) dt
\\&= \int_0^\infty \gamma^n(\nabla P_t v \cdot \frac{\nabla P_t}{P_t
  v}) dt
\\&= \int_0^\infty \gamma^n\left(\frac{||\nabla P_t v||_2^2}{P_t v}\right) dt
\\&= \int_0^\infty \gamma^n\left(\frac{e^{-2t}||P_t \nabla
    v||_2^2}{P_t v}\right) dt
\intertext{using \eqref{csstep}, which was a consequence of Cauchy-Schwartz,}
&\leq \int_0^\infty \gamma^n\left(\frac{e^{-2t}(P_t v)(P_t
    \frac{||\nabla v||^2 }{v})}{P_t v}\right) dt
\\&= \int_0^\infty \gamma^n \left(e^{-2t} P_t
    \frac{||\nabla v||^2 }{v} \right)  dt
\\&= \int_0^\infty e^{-2t} \gamma^n\left(P_t
    \frac{||\nabla v||^2 }{v} \right) dt
\intertext{using \eqref{intE},}
&= \int_0^\infty e^{-2t} \gamma^n\left(\frac{||\nabla v||^2 }{v}
\right) dt
\\&= \frac{1}{2} \gamma^n\left(\frac{||\nabla v||^2 }{v}\right)
\intertext{and since $v=f^2$}
\\&= \frac{1}{2} \gamma^n\left(\frac{||2f \nabla f||^2 }{f^2}\right)
\\&\leq \frac{1}{2} \gamma^n\left(\frac{4 f^2 ||\nabla f||^2 }{f^2}\right)
\\&= 2\gamma^n (||\nabla f||_2^2)
\end{align}
as desired. $\Box$.

\noindent\textbf{Proof of Proposition}

Define $f_t = P_t f$ and $r(t) = \gamma^n(f_t^{q(t)})$.

Since $q(0)=p$, we have
\[
||P_0 f_t||_{q(0)} = ||f_t||_p
\]
so it suffices to prove that
\[
\frac{\partial }{\partial t}||P_t f||_{q(t)} \leq 0
\]

Begin with the special case that $f$ is nonnegative.

The key fact we will use is that applying the logarithmic Sobolev
inquality to the function $f_t^{q(t)/2}$, we get
\begin{equation}\label{lsic}
\gamma^n\left(f_t^{q(t)}\log \frac{f_t^{q(t)}}{r(t)}\right) \leq
2\gamma^n(|\nabla f_t^{q(t)/2}|^2) 
= 2\gamma^n\left(\left(\frac{q(t)}{2}\right)^2 ||\nabla f_t||_2^2
\right) = \frac{q(t)^2}{2}\gamma^n(||\nabla f_t||_2^2)
\end{equation}

Note that 
\begin{equation}\label{qpe}
q'(t) = 2(p-1)e^{-2t} = 2(1+(p-1)e^{-2t} - 1)=2(q(t)-1)
\end{equation}
and
\[
\frac{\partial}{\partial f_t} = Lf_t
\]

Now
\begin{align*}
r'(t) &= \gamma^n \left( f_t^{q(t)} \frac{\partial}{\partial
    t}(q(t)\log f_t)  \right)
\\&= q'(t) \gamma^n(f_t^{q(t)} \log f_t) + q(t) \gamma^n(f_t^{q(t)-1}
Lf_t)
\intertext{using \eqref{innerproductf}}
&= q'(t) \gamma^n(f_t^{q(t)} \log f_t) + q(t) \gamma^n(\nabla f_t^{q(t)-1}
\cdot \nabla f_t)
\\&= q'(t) \gamma^n(f_t^{q(t)} \log f_t) + q(t) \gamma^n((q(t)-1)f_t^{q(t)-2}
\nabla f_t \cdot \nabla f_t)
\\&= \frac{q'(t)}{q(t)} \gamma^n(f_t^{q(t)} \log f_t^{q(t)}) -
q(t)(q(t)-1) \gamma^n(f_t^{q(t)-2}||\nabla f_t||_2^2)
\end{align*}

With this, write
\begin{align*}
\frac{\partial}{\partial t} \log ||f_t||_{q(t)} &=
\frac{\partial}{\partial t} \frac{\log r(t)}{q(t)}
\\&= \frac{-q'(t) \log r(t)}{q(t)^2} + \frac{r'(t)}{q(t)r(t)}
\\&= \frac{-q'(t) r(t)\log r(t)}{q(t)^2 r(t)} + \frac{q'(t)}{q(t)^2
  r(t)}\gamma^n(f_t^{q(t)} \log f_t^{q(t)}) - \frac{q(t)-1}{r(t)}
\gamma^n(f_t^{q(t)-2}||\nabla f_t||_2^2)
\\&= \frac{-q'(t)}{q(t)^2 r(t)}\gamma^n(f_t^{q(t)}\log r(t)) + \frac{q'(t)}{q(t)^2
  r(t)}\gamma^n(f_t^{q(t)} \log f_t^{q(t)}) - \frac{q(t)-1}{r(t)}
\gamma^n(f_t^{q(t)-2}||\nabla f_t||_2^2)
\\&=  \frac{q'(t)}{q(t)^2
  r(t)}\gamma^n\left(f_t^{q(t)} \log \frac{f_t^{q(t)}}{r(t)}\right) - \frac{q(t)-1}{r(t)}
\gamma^n(f_t^{q(t)-2}||\nabla f_t||_2^2)
\intertext{using \eqref{qpe}}
&=  \frac{q'(t)}{q(t)^2
  r(t)}\gamma^n\left(f_t^{q(t)} \log \frac{f_t^{q(t)}}{r(t)}\right) - \frac{q'(t)}{2r(t)}
\gamma^n(f_t^{q(t)-2}||\nabla f_t||_2^2)
\\&=
\frac{q'(t)}{q(t)^2
  r(t)} \left[\gamma^n\left(f_t^{q(t)} \log \frac{f_t^{q(t)}}{r(t)}\right) - \frac{q(t)^2}{2}
\gamma^n(f_t^{q(t)-2}||\nabla f_t||_2^2)\right]
\intertext{but from the logarithmic Sobolev inequality \eqref{lsic}, we know that
  the expression in the brackets is at most zero}
&\leq 0
\end{align*}

The general case follows from
\[
||P_t f||_{q(t)} = || |P_t f| ||_{q(t)} \leq || |P_t|f|| ||_{q(t)} =
|| P_t |f||_{q(t)} \leq || |f| ||_p = ||f||_p
\]

\section{References}

\noindent Chatterjee, S. (2014). \emph{Superconcentration and Related
  Topics}.  Springer.


\noindent Ledoux, M. (2005). \emph{The concentration of measure phenomenon}
(Vol. 89). American Mathematical Soc..

\end{document}