\documentclass[11pt]{article} 

% packages with special commands
\usepackage{amssymb, amsmath}
\usepackage{epsfig}
\usepackage{array}
\usepackage{ifthen}
\usepackage{color}
\usepackage{fancyhdr}
\usepackage{graphicx}
\usepackage{mathtools}
\usepackage{algorithm}
\usepackage{algpseudocode}
\definecolor{grey}{rgb}{0.5,0.5,0.5}

\begin{document}
\newcommand{\tr}{\text{tr}}
\newcommand{\E}{\textbf{E}}
\newcommand{\diag}{\text{diag}}
\newcommand{\argmax}{\text{argmax}}
\newcommand{\Cov}{\text{Cov}}
\newcommand{\Var}{\text{Var}}
\renewcommand{\thefootnote}{\fnsymbol{footnote}}

Charles Zheng CME 323 HW 1

\noindent\textbf{1.}

Checkpoint on slide 11:
\begin{verbatim}
res0: Array[Int] = Array(1, 2, 3, 4, 5, 6, 7, 8, 9)
\end{verbatim}

Checkpoint on slide 55:
\begin{verbatim}
(SparkCamp,4)
(Spark,3)
(spark,1)
(SparkSQL,1)
(../spark/bin/spark-submit,1)
\end{verbatim}

Code for slide 60:
\begin{verbatim}
val rdd1 = sc.textFile("README.md").filter(_ contains "Spark")
val rdd2 = sc.textFile("spark/docs/contributing-to-spark.md").filter(_ contains "Spark")
val wc1 = rdd1.flatMap(l => l.split(" ")).map(w => (w, 1)).reduceByKey(_ + _)
val wc2 = rdd2.flatMap(l => l.split(" ")).map(w => (w, 1)).reduceByKey(_ + _)
val joined = wc1.join(wc2)
\end{verbatim}

Checkpoint on slide 60:
\begin{verbatim}
(Spark,(3,2))
\end{verbatim}

\noindent\textbf{2.}

The mapper emits one key-value output pair for every input pair of
vertices: the output key is the \emph{sorted} vertices and the output
value indicates the direction of the edge.  The mapper takes directed
edge $\langle a, b\rangle$: if $a < b$, it emits $\langle (a, b), 1
\rangle$ and if $a > b$, it emits $\langle (b, a), 2 \rangle$.

The reducer sees all the values $v_1, \hdots, v_n$ for a given key
$(c, d)$.  If $\{1, 2\} \subseteq \{v_1,\hdots, v_n\}$ then it emits
$(c, d)$; otherwise it emits nothing.

No combiner is used; combiners are not likely to help in this problem.

\begin{algorithm}[H]
\caption{Map}
\begin{algorithmic}
\Function{Map}{$\langle a,b\rangle$}
  \If {$a < b$}
      \State Emit $\langle (a, b), 1 \rangle$
  \Else
      \State Emit $\langle (b, a), 2 \rangle$
  \EndIf
\EndFunction
\end{algorithmic}
\end{algorithm}

\begin{algorithm}[H]
\caption{Reduce}
\begin{algorithmic}
\Function{Reduce}{Key $(c, d)$, Values $\{v_1, \hdots, v_n\}$}
  \If{$n < 2$}
    \Return
  \EndIf
  \For{$i=2, \hdots, n$}
    \If{$v_i \neq v_{i-1}$}
      \State Emit $\langle c, d \rangle$
      \State \Return
    \EndIf
  \EndFor
\EndFunction
\end{algorithmic}
\end{algorithm}

\noindent\textbf{3.}

The combiner is the same as the reducer.  Let $N$ be the total number of words.

\begin{algorithm}[H]
\caption{Map}
\begin{algorithmic}
\Function{Map}{String $s$}
  \For{Word $w$ in $s$}
    \State Emit $\langle w, 1\rangle$
  \EndFor
\EndFunction
\end{algorithmic}
\end{algorithm}

\begin{algorithm}[H]
\caption{Reduce/Combine}
\begin{algorithmic}
\Function{Reduce}{Key $w$, Values $\{v_1,\hdots, v_n\}$}
  \State $s \leftarrow 0$
  \For{$i = 1, \hdots, n$}
    \State $s \leftarrow s + v_i$
  \EndFor
  \State Emit $\langle w, s\rangle$
\EndFunction
\end{algorithmic}
\end{algorithm}

\emph{Without combiners--}
The shuffle size is $N$, and the reduce takes $N \pm O(B)$ operations.

\emph{With combiners--}
After the combine step, there are at most $k$ key-value output pairs,
since ther are at most $k$ distinct keys.
The shuffle size is $kB$, and the reduce takes $kB \pm O(B)$ operations.

\newpage
\noindent\textbf{4.}

Let me briefly state the naive solution, which does not parallelize
effectively.  Run one map-reduce to count the number of elements $N$.
Next, map each input pair $\langle i, a_i \rangle$ to $N-i+1$ output
pairs $\langle i, a_i \rangle, \langle i+1, a_{i+1} \rangle, \hdots,
\langle N, a_N \rangle$.  Reduce by summing all values for a given
key.  With combiners, the shuffle size is $N$, and the number of
reduce operations is $NB$ where $B$ is the number of mappers.  The
problem with this solution is that each mapper requires $O(N)$ storage
to hold the output keys--but this is on the same order as the size of
the entire data.

A better idea is to use divide-and-conquer.  The idea is to divide the
keyset into $B$ equally-sized partitions: $P_1 = \{1,\hdots, n_1\},
P_2 = \{n_1 + 1, \hdots, n_2\}, \hdots, P_B = \{n_{B-1} + 1, \hdots
N\}$.  Accordingly define 
\[\phi(i) = b \text{ such that } i \in P_b .\]
Then define the partial sums $p_1,\hdots, p_B$ by 
\[ p_b = \sum_{i \in P_b} a_i ,\]
and define quantities
\[
u_b = \sum_{c < b} p_b .
\]
For $i = 1, \hdots, n$ define the within-partition prefix sums as
\[
t_i = \sum_{j \in P_{\phi(i)}: j \leq i} a_j
\]
Now observe that
\[
s_i = t_i + u_{\phi(i)}
\]
Therefore the procedure is as follows
\begin{enumerate}
\item (Map/Reduce 1) Count the number of keys $N$
\item (Map 2) Input: the original key-value pairs. Partition the keys
  sequentially into $B$ workers
\item (Reduce 2) Each worker $b = 1,\hdots, B$ computes $p_b$ and sends it to the driver
\item The driver computes $u_b = \sum_{c < b} p_c$ and sends $u_b$ to
  each worker $b$ for $b = 1,\hdots, B$.
\item (Reduce 3) Input: the output of step 2. Each worker $b = 1,\hdots, B$ computes $s_i = u_b + t_i$ and
  emits $\langle i, s_i \rangle$ for each $i \in P_b$.
\end{enumerate}

To formalize this procedure in the map/reduce framework we have to
designate the partition number $b$ as a key throughout steps 2-5.
In steps 2 and 5, we have $(i, a_i)$ as values.
Note that step 5 uses the same input as step 3.

\begin{algorithm}[H]
\caption{Step 2: Map 2}
\begin{algorithmic}
\State Parameters $n_1,\hdots, n_{B-1}$ determined in Step 1, and $n_B = N$.
\Function{Map}{$\langle i, a \rangle$ from original inputs}
  \For{$b \in 1,\hdots, B$}
    \If{$n_b > i$}
      \State Emit $\langle b, (i, a) \rangle$
      \State \Return
    \EndIf
  \EndFor
\EndFunction
\end{algorithmic}
\end{algorithm}

\begin{algorithm}[H]
\caption{Step 3: Reduce 2}
\begin{algorithmic}
\Function{Reduce}{Key $b$, values $(i, a)$ from step 2}
  \State $p \leftarrow 0$
  \For{$(i, a)$ in values}
    \State $p \leftarrow p + a$
  \EndFor
  \State Emit $\langle b, p \rangle$
\EndFunction
\end{algorithmic}
\end{algorithm}

\begin{algorithm}[H]
\caption{Step 5: Reduce 3}
\begin{algorithmic}
\State Parameters $u_1,\hdots u_B$ computed by driver in step 3. 
\Function{Reduce}{Key $b$, values $(i, a)$ from step 2}
  \State Sort values $(i, a)$ by $i$
  \State $s \leftarrow u_b$
  \For{Value $(i, a)$ in sorted list}
    \State $s \leftarrow s + a$
    \State Emit $\langle i, s \rangle$
  \EndFor
\EndFunction
\end{algorithmic}
\end{algorithm}

The cost of the computation is dominated by step 2, when the data is
partitioned: this requires a shuffle size of $N$.  This is followed by
a reduce in step 3 requiring $O(N/B)$ operations and $O(1)$ space.
The driver has to complete $O(B)$ operations in step 4.  Finally, each
worker has to complete $O(N/B)$ operations in step 5, requiring $O(1)$
memory.  The overall number of Map/Reduce iterations is 3, including
the initial count.


\end{document}
