\documentclass[11pt]{article} 

% transcribed by Charles Zheng

% packages with special commands
\usepackage{amssymb, amsmath}
\usepackage{epsfig}
\usepackage{array}
\usepackage{ifthen}
\usepackage{color}
\usepackage{fancyhdr}
\usepackage{graphicx}
%\usepackage{mathtools}
\definecolor{grey}{rgb}{0.5,0.5,0.5}

\begin{document}
\newcommand{\tr}{\text{tr}}
\newcommand{\E}{\textbf{E}}
\newcommand{\diag}{\text{diag}}
\newcommand{\argmax}{\text{argmax}}
\newcommand{\argmin}{\text{argmin}}
\newcommand{\Cov}{\text{Cov}}
\newcommand{\Vol}{\text{Vol}}
\pagestyle{fancy}

\title{Differential Geometric Theory of Statistics}

\author{Shun-ichi Amari\thanks{
Department of Mathematical Engineering and Instrumentation Physics, University of Tokyo, Tokyo, Japan}}

\maketitle

\tableofcontents

\section{Introduction}

            Statistics is a science which studies methods of inference, from
observed data, concerning the probabilistic structure underlying such data.
The class of all the possible probability distributions is usually too wide to
consider all its elements as candidates for the true probability distribution
from which the data were derived.  Statisticians often assume a statistical
model which is a subset of the set of all the possible probability distribut{ions,}
      and evaluate procedures of statistical inference assuming that the model
is faithful, i.e., it includes the true distribution.  It should, howver, be
remarked that a model is not necessarily faithful but is approximately so.  In
either case, it should be very important to know the shape of a statistical
model in the whole set of probability distributions.  This is the geometry of a
statistical model.  A statistical model often forms a geometrical manifold, so
that the geometry of manifolds should play an important role.  Considering that
properties of specific types of probability distributions, for example, of
Gaussian distributions, of Wiener processes, and so on, have so far been studied
in detail, it seems rather strange that only a few theories have been proposed
concerning properties of a family itself of distributions.  Here, by the proper{ties}
     of a family we mean such geometric relations as mutual distances, flatness
or curvature of the family, etc.  Obviously it is not a trivial task to define 
such geometric structures in a natural, useful and invariant manner.
            Only local properties of a statistical model are responsible for the
asymptotic theory of statistical inference.  Local properties are represented
by the geometry of the tangent spaces of the manifold.  The tangent space has a
natural Riemannian metric given by the Fisher information matrix in the regular
case.  It represents only a local property of the model, because the tangent
space is nothing but local linearization of the model manifold.  In order to
obtain larger-scale properties, one needs to define mutual relations of the two
different tangent spaces at two neighboring points in the model.  This can be
done by defining a one-to-one affine correspondence between two tangent spaces,
which is called an affine connection in differential geometry.  By an affine
connection, one can consider local properties around each point beyond the
linear approximation.  The curvature of a model can be obtained by the use of
this connection.  It is clear that such a differential-geometrical concept pro{vides}
      a tool convenient for studying higher-order asymptotic properties of
inference.  However, by connecting local tangent spaces further, one can obtain
global relations.  Hence, the validity of the differential-geometrical method is
not limited within the framework of asymptotic theory.

\section{Geometrical Structure of Statistical Models}

\section{Higher-Order Asymptotic Theory of Statistical Inference in Curved Exponential Family}

\section{Information, Sufficiency and Ancillarity Higher Order Theory}

\section{Fibre-Bundle Theory of Statistical Models}

\section{Estimation of Structural Parameter in the Presence of Infinitely Many Nuisance Parameters}

\section{Parametric Models of Stationary Gaussian Time Series}

\section{References}

\end{document}
