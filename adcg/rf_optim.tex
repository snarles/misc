\documentclass[11pt]{article} 

% packages with special commands
\usepackage{amssymb, amsmath}
\usepackage{epsfig}
\usepackage{array}
\usepackage{ifthen}
\usepackage{color}
%\usepackage{fancyhdr}
\usepackage{graphicx}
\usepackage{indentfirst}
\usepackage{caption}
%\usepackage{mathtools}
\definecolor{grey}{rgb}{0.5,0.5,0.5}

\begin{document}
\newcommand{\tr}{\text{tr}}
\newcommand{\E}{\textbf{E}}
\newcommand{\diag}{\text{diag}}
\newcommand{\argmax}{\text{argmax}}
\newcommand{\argmin}{\text{argmin}}
\newcommand{\Cov}{\text{Cov}}
\newcommand{\Var}{\text{Var}}
\newcommand{\Vol}{\text{Vol}}
\newcommand{\HH}{\boldsymbol{H}}
%\pagestyle{fancy}

\title{Random Fields and Optimization}

\author{Charles Zheng and ?}

\maketitle

\begin{abstract}
One of the most common approaches for minimizing a function $f(x)$
over a compact domain is to apply a deterministic local descent
method, although such an approach is highly likely to yield a local
minimum.  Using techniques borrowed from the optimal transport
literature, we propose a new method for characterizing the probability
of convergence to the global minimum from a random starting point.  In
limit of infinitely small step sizes, the local descent iterates
$x^{(0)}, \hdots, $ trace out a continuous trajectory $x(t)$, where
$t$ is the limit of the iteration number times the step size.
Consequently, the domains of attraction for local minima are simply
connected and partition the domain, and the probability of convergence
to the global minimum is equal to the mass of the initialization
distribution contained in the domains of attraction of global minima.
One can therefore achieve bounds on the probability of global
convergence by bounding the \emph{radius} or \emph{diameter} of the
domain of attraction.  While such computations may be generally
intractable, here we obtain average-case bounds for such quantities
for $f$ drawn from a random ensemble, e.g., a Gaussian process.  This
produces convergence probabilities for a variety of applications: for
instance, sparse inverse problems.
\end{abstract}

\section{Introduction}
Suppose we are given a function $f: \mathcal{X} \to \mathbb{R}$, where $\mathcal{X}$ is a compact domain in $\mathbb{R}^p$,
and we have access to the gradient $\nabla f$, and possibly the Hessian $H f$.
Then we can solve the optimization problem
\[
\text{minimize} f(x)
\]
by applying a local descent method, such as gradient descent or
Newton's method.  In general, such a method can be characterized by a
direction mapping $g: \mathcal{X} \to \mathbb{R}^p$, and produces
iterates $x^{(1)}, x^{(2)},\hdots$ given by
\[x^{(k+1)} = x^{(k)} - \epsilon g(x^{(k)}),\]
defining $x^{(0)}$ to be the starting point $x_0$.  In stochastic
descent methods, $g$ can be a random mapping, but throughout this
paper we limit our scope to deterministic methods.  Two well-known
examples are gradient descent,
\[
g(x) = \nabla f(x)
\]
and Newton's method,
\[
g(x) = (H f(x))^{-1} \nabla f(x)
\]
Having chosen a step size, it follows that the iterates $x^{(k)}$ are
each a function of the starting point $x^{(0)}$.

In limit of infinitely small step size, $\epsilon \to 0$, the local
descent iterates $x^{(1)}(x^{(0)}), x^{(2)}(x^{(0)}), \hdots, $ trace
out a continuous trajectory $x(x^{(0)}, t)$, defined by
\[
x_t(x_0) = \lim_{\epsilon \to 0} x^{(\rfloor t/\epsilon \lfloor)}(x_0).
\]
The \emph{length} of the trajectory is defined as
\[
S(x_0) = \int_0^\infty || \nabla x_t(x_0)|| dt.
\]
Therefore, a sufficient condition for \emph{non-convergence} to a global minimum $x^*$ is that
\[
S(x_0) \geq ||x_0 - x^*||.
\]
Hence, one can get a sense of the difficulty of the problem by
estimating the trajectory lengths.  Intuitively, if most trajectory
lengths are small, this indicates that the problem is riddled with
local minima.

Suppose we initialize the local descent at a random point $x^{(0)}$,
where the law of $x^{(0)}$ has density $\mu^0(x)$ with respect to
Lesbegue measure.  Then the marginal density $\mu^t(x)$ of the current
iterate at time $t$ is given by the \emph{continuity equation}:
\[
\frac{\partial}{\partial t} \mu^t(x) = -\nabla \cdot (f'(x) \mu^t(x)).
\]

Considering the densities $\mu^t$ allows one to relate local
properties, such as trajectory length, to statistics of the local
minima.  For instance, one can compute the \emph{average path length} via the integral
\[
\E_{\mu^0}[S(x_0)] = \int_0^\infty \int_{\mathcal{X}} ||g(x)|| \mu^t(x) dx dt.
\]

\emph{Question:} Can we obtain bounds for the average path length, if,
for example, $f$ is a Gaussian process?

\section{Example: Sparse recovery in the white-noise limit}

In numerous applications, one is interested in reconstructing the
locations and parameters of multiple signal sources from noisy
observations.  For instance, in super-resolution imaging, one uses a
microscope to obtain a 2D image of a number of fluorescing point sources,
and the goal is to recover the locations of the point sources on the
slide.

Consider a sparse recovery problem with parameters $\theta \in
\mathbb{R}^d$ corresponding to locations of point sources, where the
signal a source at $\theta$ is given by
\[
\psi_\theta(x) = \psi(x - \theta),
\]
where $\psi: \mathbb{R}^p \to \mathbb{R}^+$ is a nonnegative kernel
function.  

Such a sparse inverse problem is described by a known,
or unknown number of sources $K$, a vector of parameters $\theta_i \in
\mathbb{R}^p$ describing the $i$th source (e.g. location in space),
and a positive real weight $w_i$ giving amplitude of the signal from
the $i$th source.  The signal from a single source $\theta$ is a
function in $\mathbb{R}^d$, denoted by $\psi_\theta$,
and the combined signal from all $K$ sources is given by the function
\[
m(x) = \sum_{i=1}^K w_i \psi_{\theta_i}(x).
\]
In the case of super-resolution imaging, for example, the parameter
$\theta \in \mathbb{R}^2$ gives the location of the point source on
the slide, $w_i$ gives the intensity of the fluorescence, and
$\psi_{\theta}(x)$ is a symmetric kernel function centered at $\theta$,
given by the point-spread function of the lens.

Given measurement locations $x_1,\hdots, x_N$, we observe signals
\[
y_i = m(x_i) + \epsilon_i,
\]
where $\epsilon_i$ are indedependent with mean $0$ and variance $\sigma^2$.

Having observed $y_1,\hdots, y_N$, our goal is to estimate the number
of sources $K$ (if $K$ is unknown), and to recover the locations
$\theta_1,\hdots, \theta_K$.
A natural approach is to minimize the objective function
\[
\text{minimize} \sum_{i=1}^n \left\|y_i - \sum_{j=1}^K w_j \psi_{\theta_j}(x_i) \right\|^2.
\]

\section{References}

\begin{itemize}
\item Ambrosio, Luigi, Nicola Gigli, and Giuseppe Savaré. Gradient flows: in metric spaces and in the space of probability measures. Springer Science \& Business Media, 2008.
\item Morgenshtern, Veniamin I., and Emmanuel J. Candes. "Super-Resolution of Positive Sources: the Discrete Setup." arXiv preprint arXiv:1504.00717 (2015).
\end{itemize}



\end{document}
