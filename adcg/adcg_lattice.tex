\documentclass[11pt]{article} 

% packages with special commands
\usepackage{amssymb, amsmath}
\usepackage{epsfig}
\usepackage{array}
\usepackage{ifthen}
\usepackage{color}
%\usepackage{fancyhdr}
\usepackage{graphicx}
\usepackage{indentfirst}
\usepackage{caption}
%\usepackage{mathtools}
\definecolor{grey}{rgb}{0.5,0.5,0.5}

\begin{document}
\newcommand{\tr}{\text{tr}}
\newcommand{\E}{\textbf{E}}
\newcommand{\diag}{\text{diag}}
\newcommand{\argmax}{\text{argmax}}
\newcommand{\argmin}{\text{argmin}}
\newcommand{\Cov}{\text{Cov}}
\newcommand{\Var}{\text{Var}}
\newcommand{\Vol}{\text{Vol}}
\newcommand{\HH}{\boldsymbol{H}}
%\pagestyle{fancy}

\title{When does Alternating Descent Conditional Gradient work better than non-convex optimization with random restarts?}

\author{Charles Zheng and ?}

\maketitle

\begin{abstract}
Alternating Descent Conditional Gradient (ADCG) is a method for
solving sparse inverse problems which combines nonconvex and convex
optimization techniques.  Although ADCG is observed to achieve
superior performance to alternative approaches, little is known about
its performance due to its non-convex subroutines.  In this work, we
consider a sparse inverse problem with i.i.d. noise on a
$d$-dimensional lattice, and given a general nonlinear optimization
subroutine, compare the performance of ADCG equipped with the given
subroutine versus the approach of simply applying the subroutine with
random intializations.  We establish a regime where the size of the
lattice scales with the number of sources as well as the amplitude of
each source, in which ADCG converges to the global minimum with fixed
number of calls while the probability of reaching the global minimum
under a random intitialization goes to zero.
\end{abstract}

\section{Introduction}

In numerous applications, one is interested in reconstructing the
locations and parameters of multiple signal sources from noisy
observations.  For instance, in super-resolution imaging, one uses a
microscope to obtain a 2D image of a number of fluorescing point sources,
and the goal is to recover the locations of the point sources on the
slide.

Such a sparse inverse problem is described by a known,
or unknown number of sources $K$, a vector of parameters $\theta_i \in
\mathbb{R}^p$ describing the $i$th source (e.g. location in space),
and a positive real weight $w_i$ giving amplitude of the signal from
the $i$th source.  The signal from a single source $\theta$ is a
function in $\mathbb{R}^d$, denoted by $\psi_\theta$,
and the combined signal from all $K$ sources is given by the function
\[
\mu(x) = \sum_{i=1}^K w_i \psi_{\theta_i}(x).
\]
In the case of super-resolution imaging, for example, the parameter
$\theta \in \mathbb{R}^2$ gives the location of the point source on
the slide, $w_i$ gives the intensity of the fluorescence, and
$\psi_{\theta}(x)$ is a symmetric kernel function centered at $\theta$,
given by the point-spread function of the lens.

Given measurement locations $x_1,\hdots, x_N$, we observe signals
\[
y_i \sim F(\mu(x_i)),
\]
where $\{F(\mu)\}_{\mu \in \mathbb{R}}$ is a family of parametric distributions.
For instance, a simple case is $F(\mu) = N(\mu, \sigma^2)$, corresponding to the familiar Gaussian error model
\[
y_i = \mu(x_i) + \epsilon_i, \ \epsilon_i \sim N(0, \sigma^2).
\]
Very shortly, we will assume that $F(\mu)$ has a density $f_\mu(y)$ on $\mathbb{R}$,
and hence a negative log-likelihood function
\[
\ell(y; \mu) = -\log(f_\mu(y))
\]

Having observed $y_1,\hdots, y_N$, our goal is to estimate the number
of sources $K$ (if $K$ is unknown), and to recover the locations
$\theta_1,\hdots, \theta_K$.
A natural approach is to minimize the objective function
\[
\text{minimize} \sum_{i=1}^n \ell\left(y_i; \sum_{j=1}^K w_j \psi_{\theta_j}(x_i) \right)
\]
subject to some sparsity constraint, where $\ell$ is the
log-likelihood function for $\{F(\mu)\}$.  
The objective function is a function of a set of weight-parameter pairs $\{w_i, \theta_i\}$,
which as Boyd et al. noted describe a positive measure on the parameter space
\[
\sum_{i=1}^K w_i \delta_{\theta_i}.
\]
Since we are concerned with computational aspects in this work, we
will generally work with the weight-parameter pairs represented as a
set of tuples $(w_i,\theta_i)$ rather than as a measure, though we
will borrow from the terminology and refer to a weight-parameter pair
$(w_i,\theta_i)$ as an \emph{atom}, $w_i$ as the weight or amplitude
of the $i$th atom, and $\theta_i$ as the location of the atom.

However, throughout the literature, a variety of methods have been
proposed for choosing the sparsity constraint and for minimizing the
objective function.  We describe three main categories of methods:
\begin{itemize}
\item Using nonlinear optimization (e.g. gradient descent or Newton's
  method) to minimize the objective function subject to a constraint
  on the putative number of sources $K$.
\item Discretizing the parameter space by choosing candidate
  parameters $\theta_1,\hdots, \theta_m \in \mathbb{R}^p$, then finding the optimal weights
\[
\text{minimize}_w \sum_{i=1}^n \ell\left(y_i; \sum_{j=1}^m w_j \psi_{\theta_j}(x_i) \right),
\]
possibly subject to an $L_1$-norm constraint or penalty on the weights $w$.
Note that the discretized problem is \emph{convex} if $\ell$ is convex.
\item Alternating Descent Conditional Gradient.  (To be described below.)
\end{itemize}

Nonlinear optimization is not guaranteed to achieve the global minimum
of the objective function, hence a usual approach is to run nonlinear
optimization multiple times with random starting conditions.  As a
result, it is often quite costly to get a good solution of the
optimization problem using nonlinear optimization.

In contrast, when $\ell$ is convex, it is possible to
deterministically approximate the global minimum with the
discretization approach, by choosing a suitably fine discretization
and then applying convex optimization to solve the discretized
objective function.  However, one is limited to using convex
constraints, which excludes the possibility of solving the
optimization problem subject to a constraint or penalty on the number
of sources $K$, which is equal to the $L_0$ norm of the weights.
Instead, here one typically places a constraint or penalty on the
$L_1$ norm of the weights, since the $L_1$-norm is the tightest convex
relaxation of the $L_0$ norm.  Intuitively, one expects the $L_1$
convex relaxation to yield a worse solution than the $L_0$ constrained
problem; while plenty of theoretical results (Morgenshtern, Candes,
etc.) establish the statistical properties of the estimators resulting
from $L_1$ minimization, little is known about the comparative
performance of $L_0$-constrained minimization, even supposing that the
global minima are achieved.  In particular, it is not possible to
apply the sparse recovery results of Donoho et al., since the design
matrix in our problem is typically highly collinear and hence violates
the usual $L_1$ support recovery conditions.  It is worth mentioning
the work by Slawski (2012), which introduces a framework for studying
the sparse recovery problem given such highly correlated design
matrices, and which also provides results on denoising.

Alternating Descent Conditional Gradient (Boyd et al.) combines the
convex and nonconvex approaches.  It is shown to have guaranteed
performance to the global minimizer, subject to \emph{convex} sparsity
constraints.  The algorithm is defined with reference to a gradient
subroutine $\tau$ and a nonlinear descent subroutine $\nu$.  The
gradient subroutine is given residuals $r_1,\hdots, r_N$ as input, and
and outputs the parameter $\theta$ whose signal maximizes the inner
product with the gradient of the loss with respect to the residuals:
\[
\tau(r_1,\hdots, r_n) = \argmax_\theta \sum_{i=1}^n \psi_\theta(x_i) \dot{\ell}(r_i).
\]
The descent subroutine $\nu$ is given a set of atoms
$\{(w_i,\theta_i)\}_{i=1}^K$ as input, and applies nonlinear
optimization to the objective function, starting from the input
parameters.  The output will be a set of $K$ atoms
which constitute a local minimum of the objective function, and with
the property that the output set has an equal or lower value of the
objective function than the input set.

The ADCG algorithm iteratively updates a set of atoms $\{(w_i,
\theta_i)\}$, whose size may change from iteration to iteration.  The
parameter set is intitialized as the empty set.  In each iteration,
the gradient subroutine is applied to the current residuals to yield a
new parameter $\theta$, which defines the location of a new atom $(0,
\theta)$ that is added to the set of atoms.  The convex reweighting
step minimizes the objective function fixing the parameters
$\theta_i$.  Having updated the weights $w_i$, one optionally prunes
all atoms with zero weight, Finally, one applies the descent
subroutine $\nu$ to jointly update the weights and locations of the
atoms.

In each iteration, the atoms grows by at most one, due to newly atom
from the gradient subroutine.  Unless the algorithm has already
converged, the newly added atom is sure to acquire a positive weight
after the convex reweighting step; hence it will not be pruned.
However, one or more of the atoms from previous iterates may be
pruned; hence the set of atoms may experience a net decrease in size.

While ADCG is guaranteed to optimize the global minimimum under a
convex constraint, one can also consider using the algorithm to
optimize the $L_0$ constraint.  Given a constraint on the number of
atoms $K$, one runs ADCG until the number of atoms reaches $K$.  Due
to the descent subroutine, the resulting set of atoms is guaranteed to
be a local optimum of the objective function.  Utilizes this way, ADCG
can be thought of as an intelligent way to iteratively build up a good
intitialization for the nonlinear descent step in the final iteration.
Additionally, if the number of sources $K$ is unknown, one can select
for each possible $K=1,2,\hdots$, the ADCG iterate with the best
objective function.  This allows one to efficiently obtain solutions
for the $L_0$-constrained problem for each candidate value of $K$.

In this paper, we will focus on the problem of solving the
optimization problem subject to an $L_0$ constraint.  We will leave
the analysis of the discretization approach to future
work\begin{footnote}{While methods for converting an $L_1$-sparse
  solution to an $L_0$-sparse solution exist in the literature
  (e.g. ``peak-finding'' in diffusion-weighted imaging), such methods
  are usually application-specific.}\end{footnote}, and concentrate on
comparison of ADCG and the ``naive'' approach of applying nonlinear
descent with random starting conditions.  Since the same nonlinear
descent algorithm can be used in both ADCG and the random restart
approach, this allows the two approaches to be compared ``on equal
footing'': namely, we measure the complexity of each approach by the
number of calls to the nonlinear descent algorithm $\nu$.  Since the
nonlinear descent algorithm is usually the bottleneck in practice,
this provides a reasonable estimate of the true computational cost.

In the following section, we present our model and the particular
variants of ADCG and nonlinear descent to be studied.

\section{Setup}\label{section:Setup}

Consider a sparse recovery problem with parameters $\theta \in
\mathbb{R}^d$ corresponding to locations of point sources, where the
signal a source at $\theta$ is given by
\[
\psi_\theta(x) = \psi(x - \theta),
\]
where $\psi: \mathbb{R}^p \to \mathbb{R}^+$ is a nonnegative kernel
function.  

Let the true atoms be given by the set $\{(w_i^0,
\theta_i^0)\}_{i=1}^{K^0}$, such that $w_i^0 > 0$.  Furthermore, let
us assume that $\theta_i$ are known \emph{a priori} to lie in a
hypercube $[-T, T]^d$.  We observe data $y_z$ for each measurement
point $z$ in the lattice $Z = \{-n,..., n\}^d \in \mathbb{Z}^d$, given by
\[
y_z = \epsilon_z + \sum_{i=1}^{K^0} w_i^0 \psi(z - \theta_i^0)
\]
where $\epsilon_z$ are identically and independently distributed
according to a distribution $F$, with zero mean and unit variance.

In our proof we will also make the following additional assumptions:
\begin{itemize}
\item[A1.] We assume that $\psi$ is twice-differentiable and has a
bounded support in the sense that there exists a bandwidth $h > 0$
such that $\psi(x) = 0$ for all $x \in \mathbb{R}^p$ with $||x|| > h$.
\item[A2.] $F$ has bounded support, i.e. there exists $c < \infty$ so that $\Pr[|y| < c] = 1$ for $y \sim F$.
\item[A3.] $\psi$ and its first derivatives are bounded in $L_\infty$ norm by some $M < \infty$.
\item[A4.] \[
\inf_\theta \sum_z \phi(z - \theta)^2  = B^{-1} > 0.
\]
\item[A5.] $n >> T$, i.e. the size of the data lattice is much larger than
  the bounding box for the true parameters.  This assumption simply
  allows us to ignore boundary issues.
\end{itemize}
All of the above assumptions are made purely for convenience, and any
number of them could be relaxed.

Consider the problem of minimizing the objective under $L_2$ loss, given by
\[
\mathcal{L}(\{w_i, \theta_i\}) = \frac{1}{2}\sum_{z \in Z} \left\|y_z - \sum_{j=1}^K w_j \psi(z - \theta) \right\|^2
\]
where $K$, the number of atoms, is fixed.

We compare the following two approaches:
\begin{itemize}
\item \emph{Random restarts.} 
For a fixed number of iterations $k = 1,\hdots, M$,
choose a random starting condition $\{(0, \theta_i^{(k, 0)})\}_{i=1}^K$
by drawing $\theta_i$ i.i.d. from the uniform distribution on $[-T,T]^d$.
Apply nonlinear subroutine $\nu$ with the starting condition
to obtain local minimum $\{(w_i^{(k, 1)}, \theta_i^{(k, 1)})\}_{i=1}^K$,
and let $\mathcal{L}^{(k)}$ denote the value of the objective function.
After $M$ such iterations, return $\{(w_i^{(k)}, \theta_i^{(k)})\}$ with the smallest $o^{(k)}$.
\item \emph{ADCG without pruning.}
Run the ADCG algorithm without pruning for $K$ iterations, then return the final set of atoms.
In the gradient step of ADCG, limit the search to $\theta \in [-T, T]^d$.
\end{itemize}

For both approaches, we consider \emph{gradient descent} with a fixed
step size $\epsilon_{grad}$ and a fixed number of steps $L_{grad}$, which we
describe explicitly in the following section.

We use the following criterion for evaluating the algorithms.
Assume that the true number of sources $K_0$ is known, and $K = K_0$.
Then fixing $\Delta > 0$, we say that the estimated parameters $\theta_1,\hdots, \theta_K$
\emph{recover the support within} $\Delta$ if
\[
\max_i \min_j ||\theta_i^0 - \theta_j|| < \Delta.
\]

\section{Gradient descent}

Under the setup described in \ref{section:Setup},
the gradient of the objective at a particular a set of atoms $\{(w_i, \theta_i)\}_{i=1}^K$
is given by
\[
\frac{\partial \mathcal{L}}{\partial w_i} = -\sum_{z \in Z} r_z(\{(w_j, \theta_j)\}_{j=1}^K) \psi(z - \theta_i)
\]
\[
\frac{\partial \mathcal{L}}{\partial \theta_{ij}} = -w_i\sum_{z \in Z} r_z(\{(w_j, \theta_j)\}_{j=1}^K) \psi_j(z - \theta_i)
\]
where $\psi_j$ denotes the $j$th partial derivative of $\psi(\theta)$, and $r_z$ denotes the residual at $z$:
\[
r_z(\{(w_i, \theta_i)\}_{i=1}^K) = y_z - \sum_{i=1}^K w_i \psi(z - \theta).
\]
Given a set of atoms $\{(w_i, \theta_i)\}_{i=1}^K$, gradient descent produces iterates
$\{(w_i^{(k)}, \theta_i^{(k)})\}_{i=1}^K$ for $k = 1,\hdots, L_{grad}$,
defined recusively by
\[
w_i^{(k + 1)} = \left[w_i^{(k)} - \epsilon_{grad} \frac{\partial \mathcal{L}}{\partial w_i}(\{(w_j^{(k)}, \theta_j^{(k)})\}_{j=1}^K)\right]_+
\]
\[
\theta_{im}^{(k + 1)} = \theta_{ij}^{(k)} - \epsilon_{grad} \frac{\partial \mathcal{L}}{\partial \theta_{im}}(\{(w_j^{(k)}, \theta_j^{(k)})\}_{j=1}^K)
\]
for $i = 1,\hdots, K$ and $m=1,\hdots, d$, and where the initial iterate $\{(w_i^{(0)},
\theta_i^{(0)})\}_{i=1}^K$ is given by the the input set.

The following notation will be used in the proofs.
Define $Z_{\theta, r}$ to be the set of points $z$ in the lattice $Z$ with $||z - \theta|| < r$.
Note that by assumption A1, it follows that
\[
\sum_{z \in Z} y_z \phi(z - \theta) = \sum_{z \in Z_{\theta, h}} y_z \phi(z - \theta).
\]
Also note that the number of elements in $Z_{\theta, r}$ is at most $(2r + 1)^d$.

\subsection{Non-recovery regime}

First we present a deterministic sufficient condition so that gradient descent fails to recover the support within $\Delta$.

\textbf{Theorem 1.}
\emph{
Assume conditions A1-A5.
Define
\[S = 2\max\{2h, \Delta, L_{grad}\epsilon_{grad} c^2 H^2M^2B (1 + H MB)\sqrt{d} \},\]
where
\[
H = \max_\theta |Z_{\theta, h}| < (2h+1)^d
\]
Let $\theta_1^0,\hdots, \theta_K^0$ denote the true parameters
and $\theta_1,\hdots, \theta_K$ denote the starting parameters for gradient descent.
Suppose for all $i=1,\hdots, K$ and $j = 1,\hdots, K$ 
\[
||\theta_i - \theta_j^0|| > 2S
\]
and if $i \neq j$,
\[
||\theta_i - \theta_j|| > 2S
\]
Then gradient descent fails to recover the support within $\Delta$.
}

\textbf{Proof.}
We claim that for iterations $k = 1,\hdots, L_{grad}$, and for $i = 1,\hdots, K$, we have
\[
w_i^{(k)} < c HMB,
\]
and also that
\[
||\theta^{(k)}_i - \theta^{(k-1)}_i|| < \epsilon_{grad} c^2 H^2M^2B (1 + H MB)\sqrt{d}
\]
The claim implies that
\[
||\theta_i^{(L_{grad})} - \theta_j^0|| \geq ||\theta_i - \theta_j^0|| - \sum_{k=1}^{L_{grad}} ||\theta^{(k)} - \theta^{(k-1)}|| \geq S \geq \Delta
\]
for all $i, j = 1,\hdots, K$,
which implies that gradient descent fails to recover the support within $\Delta$.

We prove the claim by induction on $k$.  For $k = 0$ the claim clearly
holds.  Assume the induction hypothesis holds for iterations $j =
1,\hdots, k$.  Then for all $i, j = 1,\hdots, K$ it follows that
\[
||\theta_i^{(k)} - \theta_j^0|| \geq ||\theta_i - \theta_j^0|| - \sum_{k=1}^{L_{grad}} ||\theta^{(k)} - \theta^{(k-1)}|| \geq S \geq 2h
\]
This means that for any $i, j$, the sets $Z_{\theta_i^{(k)}, h}$ and $Z_{\theta_j^{(0)}, h}$ are disjoint,
i.e. $Z_{\theta_i^{(k)}, h}$ only consist of noise variates.
It follows that
\[
\max_{i=1}^K \sup_{z \in Z_{\theta_i^{(k)}, h}} |y_z| < c.
\]
Therefore,
\begin{align*}
w^{(k+1)}_i &= \left[w^{(k)}_i + \epsilon_{grad} \sum_{z \in Z} (y_z - w^{(k)}_i \phi(z - \theta^{(k)}_i)) \phi(z - \theta^{(k)}_i)\right]_+
\\&= \left[w^{(k)}_i + \epsilon_{grad} \sum_{z \in Z_{\theta_i^{(k)}}, h} (y_z - w^{(k)}_i \phi(z - \theta^{(k)}_i)) \phi(z - \theta^{(k)}_i)\right]_+
\\&= \left[w^{(k)}_i\left( 1- \epsilon_{grad} \sum_z \phi(z - \theta^{(k)}_i))^2 \right) + \epsilon_{grad} \sum_{z \in Z_{\theta_i^{(k)}}, h} y_z \phi(z - \theta^{(k)}_i)\right]_+
\\&\leq w^{(k)}_i (1 - \epsilon_{grad} B^{-1}) + \epsilon_{grad} H c M
\end{align*}
which, by the induction hypothesis yields
\[
w_i^{(k+1)} < c HMB.
\]

Similarly, we have
\begin{align*}
|\theta^{(k+1)}_{ij}-\theta^{(k)}_{ij}| &=  \left |\epsilon_{grad} w^{(k)}_i \sum_{z \in Z} (y_z - w^{(k)}_i \phi(z - \theta^{(k)}_i)) \phi_j(z - \theta^{(k)}_i)\right |
\\& \leq \epsilon_{grad} c HMB H (c + c H MB) M = \epsilon_{grad} c^2 H^2M^2B (1 + H MB)
\end{align*}
implying
\[
||\theta^{(k+1)} - \theta^{(k)}|| \leq \epsilon_{grad} c^2 H^2M^2B (1 + H MB)\sqrt{d}.
\]

Having completed the induction, we conclude the proof. $\Box$.

A trivial probabilistic argument then yields the stochastic version of the non-recovery result,
given a sequence of problems where $K^0$ remains fixed while $T$ and $n$ go to infinity.

\textbf{Corollary.}  \emph{ Assume conditions A1-A5.  Given fixed
  problem parameters $\Delta$ and $K$, for any sequence $T_i \to
  \infty$, $i = 1, 2, \hdots, $, with $n_i \geq T_i + S$ for $S$
  defined in Theorem 1, and true parameters $\{\theta_1^{0, i},
  \hdots, \theta_K^{0, i}\} \subset [-T_i, T_i]^d$, define the
  probability $p_i$ as the probability that uniformly distributed
  $\theta_1,\hdots, \theta_j$ satisfy the conditions of Theorem 1.
  Then
\[
\lim_{i \to \infty} p_i = 1,
\]
i.e. gradient descent fails to recover the support with $\Delta$ with
high probability.}



\end{document}
