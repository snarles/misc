\documentclass[11pt]{article} 

% packages with special commands
\usepackage{amssymb, amsmath}
\usepackage{epsfig}
\usepackage{array}
\usepackage{ifthen}
\usepackage{color}
\usepackage{fancyhdr}
\usepackage{graphicx}
\usepackage{mathtools}
\usepackage{amsthm}
\definecolor{grey}{rgb}{0.5,0.5,0.5}

\begin{document}
\newcommand{\tr}{\text{tr}}
\newcommand{\E}{\textbf{E}}
\newcommand{\diag}{\text{diag}}
\newcommand{\argmax}{\text{argmax}}
\newcommand{\Cov}{\text{Cov}}
\newcommand{\Var}{\text{Var}}
\newtheorem{theorem}{Theorem}

Charles Zheng

\section{Background}

\begin{theorem}
\label{bound}
Define $\phi(x) = \frac{1}{\sqrt{2\pi}}e^{-x^2/2}$ and
\[
\Psi(x) = \int_x^\infty \phi(t) dt
\]
for $x \in \mathbb{R}$.
Then
\[
\left(\frac{1}{x^3}-\frac{1}{x}\right) \phi(x) \leq \Psi(x) \leq \frac{1}{x}\phi(x)
\]
\end{theorem}

\begin{theorem}
\label{bt}
\textbf{(Borell-TIS inequality)}
Let $f_t$ be a gaussian process such that $\mathbb{E}[f_t] = 0$
Then on any measurable set $D$, and $u > 0$, 
\[
\mathbb{P}[\sup_D f_t > u + \mathbb{E}[\sup_D f_t]] \leq \exp(-u^2/(2\sigma_{max}^2))
\]
where
\[
\sigma_{max} = \sup_D \E[f_t^2]
\]
\end{theorem}

\begin{theorem}\label{slepian}
\textbf{(Slepian's inequality)}
If $f$ and $g$ are as bounded, centered gaussian processes, and
\[
\mathbb{E}[(f_t-f_s)^2] \leq \mathbb{E}[(g_t-g_s)^2]
\]
then
\[
\mathbb{P}[\sup_{t \in D} f_t  > u] \leq \mathbb{P}[\sup_{t \in D} g_t  > u]
\]
\end{theorem}

\section{Supremum of an isotropic GP}

\begin{theorem}
Let $f_t$ be a gaussian process on $\mathbb{R}$, with $\Cov(f_t,f_u)
= C(t-u)$, where $C(0) = 1$, and $C(t) \to 0$ as $||t|| \to \infty$.
Then supposing $f_t$ is bounded on $[0,1]$,
\[
\mathbb{P}\left(\liminf_{T \to \infty}\frac{\sup_{[0,T]}
    f_t}{\sqrt{2 \log(T)}} \geq 1
\right) = 1
\]
\end{theorem}

\noindent\textbf{Proof.}

It suffices to prove
\[
\mathbb{P}\left( 1-\varepsilon \leq \liminf_{T \to \infty}\frac{\sup_{[0,T]}
    f_t}{\sqrt{2 \log(T)}}\right) = 1
\]
for arbitrary $\varepsilon \in (0,1)$.

Take $\varepsilon \in (0,1)$.

Find $\tau > 0$ such that $C(\tau) <
\frac{\varepsilon}{2-\varepsilon}$ and find $T_0 > 0$ such that $T > \max\{ 2\tau,\frac{2-\varepsilon}{\varepsilon} \log(2\tau), e^{\frac{1-C(\tau)}{(1-\varepsilon)^2}}\}$.  For each of $n =
1,\hdots,$, let $T = T_0 + n$, and let let $m = \lfloor
\frac{T+1}{\tau} \rfloor$.  Define $t_k = k\tau$ for $k = 1,\hdots,m$.
Let $Z_1,\hdots, Z_m$ be iid $N(0,1-C(\tau))$.  For $i \neq j$, we have
\[
\mathbb{E}[(Z_i - Z_j)^2] = 2(1-C(\tau)) \leq 2(1-C((i-j)\tau)) = \mathbb{E}[(f_{t_i}-f_{t_j})^2]
\]

Hence by Slepian's inequality,
\[
\mathbb{P}(\sup_{t \in [0,T]} f_t > u) \geq
\mathbb{P}(\max_{i \in \{1,\hdots,m\}} f_{t_i} > u) \geq
\mathbb{P}(\max_{i \in \{1,\hdots,m\}} Z_i > u)\]
for all $u > 0$.
Thus, taking $u = (1-\varepsilon)\sqrt{2\log(T+1)}$
so that
\[
u \leq 
\left(1-\frac{\varepsilon}{2}\right) \sqrt{2(1-C(\tau)) \log\left(\frac{T}{\tau}-1\right)}
\]
and
\[
\frac{\sqrt{1-C(\tau)}}{u} - \frac{(1-C(\tau))^{3/2}}{u^3} \leq \frac{\sqrt{1-C(\tau)}}{2u}
\]
we have
\begin{align}
\mathbb{P}(\sup_{t \in [0,T]} f_t < \sqrt{2\log (T-1)}(1-\epsilon))
&\leq \mathbb{P}(\max_{i \in \{1,\hdots,m\}} Z_i < u)
\\&= \left(1-\Psi\left(\frac{u}{\sqrt{1-C(\tau)}}\right)\right)^m
\\&\leq \left(1-\left(\frac{\sqrt{1-C(\tau)}}{u} - \frac{(1-C(\tau))^{3/2}}{u^3}\right)\phi\left(\frac{u}{\sqrt{1-C(\tau)}}\right)\right)^m
\\&\leq \left(1-\left(\frac{\sqrt{1-C(\tau)}}{2u}\right)\phi\left(\frac{u}{\sqrt{1-C(\tau)}}\right)\right)^m
\\& \leq \exp\left(-m\left(\frac{\sqrt{1-C(\tau)}}{2u}\right)\phi\left(\frac{u}{\sqrt{1-C(\tau)}}\right)\right) 
\\& \leq \exp\left(-m\left(\frac{\sqrt{1-C(\tau)}}{2u}\right)\phi\left(\left(1-\frac{\varepsilon}{2}\right)\sqrt{2 \log\left(\frac{T}{\tau}-1\right)}\right)\right) 
\\& = \exp\left(-\frac{m}{\sqrt{2\pi}\left(\frac{T}{\tau}-1\right)^{(1-\frac{\varepsilon}{2})^2}}\left(\frac{\sqrt{1-C(\tau)}}{2u}\right)\right)
\\& \leq \exp\left(-\frac{\frac{T}{\tau}-1}{\sqrt{2\pi}\left(\frac{T}{\tau}-1\right)^{(1-\frac{\varepsilon}{2})^2}}\left(\frac{\sqrt{1-C(\tau)}}{2u}\right)\right)
\\& = \exp\left(-\frac{\left(\frac{T}{\tau}-1\right)^{1-(1-\frac{\varepsilon}{2})^2}}{\sqrt{2\pi}}\left(\frac{\sqrt{1-C(\tau)}}{2(1-\varepsilon)\sqrt{2\log(T-1)}}\right)\right)
\\& = \exp\left(-\frac{\left(\frac{T_0 + n}{\tau}-1\right)^{1-(1-\frac{\varepsilon}{2})^2}}{\sqrt{2\pi}}\left(\frac{\sqrt{1-C(\tau)}}{2(1-\varepsilon)\sqrt{2\log(T_0+n-1)}}\right)\right)
\end{align}
Hence
\begin{equation}\label{bcres1}
\sum_{n=1}^\infty \mathbb{P}(\sup_{t \in [0,T]} (1-\varepsilon)\sqrt{2\log(T_0+n-1)}) < \infty
\end{equation}
which by Borel-Cantelli, and the fact that
\[
\liminf_{T \to \infty} \frac{\sup_{[0,T]} f_t}{\sqrt{2 \log(T)}} \leq \liminf_{n \to \infty} \frac{\sup_{[0,T_0+n]} f_t}{\sqrt{2 \log(T_0+n-1)}}
\]
implies the result $\Box$.

\begin{theorem}
Let $f_t$ be a gaussian process on $\mathbb{R}$, with $\Cov(f_t,f_u)
= C(t-u)$, where $C(0) = 1$, and $C(t) \to 0$ as $||t|| \to \infty$.
Then supposing $f_t$ is bounded on $[0,1]$,
\[
\lim_{T \infty \infty }\mathbb{P}\left(1-\varepsilon \leq \lim_{T \to \infty}\frac{\sup_{[0,T]}
    f_t}{\sqrt{2 \log(T)}} \leq 1+\varepsilon
\right) = 1
\]
for all $\varepsilon \in (0,1)$.
\end{theorem}

\noindent\textbf{Proof.}

The lower bound follows from the previous result.
Let $\mu=\mathbb{E}[\sup_{[0,1]} f_t]$, so that
\[
\mathbb{P}[\sup_{t \in [0,1]} f_t > u] \leq e^{-(u-\mu)^2/2}
\]
for all $u > \mu$.
For any $T >0$, letting $n = \lceil T \rceil$ using union bound and isotropy, we have
\[
\mathbb{P}[\sup_{t \in [0,T]} f_t > u]  \leq \mathbb{P}[\max_{i = \{1,\hdots,n\}} \sup_{t \in [i-1,i]} f_t > u] \leq n\mathbb{P}[\sup_{t \in [0,1]} f_t > u] \leq ne^{-(u-\mu)^2/2}
\]
for all $u > \mu$.
Therefore, defining $u = (1+\varepsilon)\sqrt{2\log(T)}$, and taking $n$ large enough so that $u > \mu$, we get
\[
\mathbb{P}[\sup_{t \in [0,T]} f_t > u] \leq ne^{-(u-\mu)^2/2} = \frac{n}{(n+1)^{(1+\varepsilon)^2}}\exp\left(-\mu (1+\varepsilon)\sqrt{2\log(n+1)} - \frac{\mu^2}{2}\right)
\]
which tends to 0 as $T \to \infty$. $\Box$


\section{References}

Adler RF, Taylor J. \emph{Random Fields and Geometry}

\end{document}
