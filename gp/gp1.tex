\documentclass[11pt]{article} 

% packages with special commands
\usepackage{amssymb, amsmath}
\usepackage{epsfig}
\usepackage{array}
\usepackage{ifthen}
\usepackage{color}
\usepackage{fancyhdr}
\usepackage{graphicx}
\usepackage{mathtools}
\definecolor{grey}{rgb}{0.5,0.5,0.5}

\begin{document}
\newcommand{\tr}{\text{tr}}
\newcommand{\E}{\textbf{E}}
\newcommand{\diag}{\text{diag}}
\newcommand{\argmax}{\text{argmax}}
\newcommand{\Cov}{\text{Cov}}
%\newcommand{\Var}{\text{Var}}

Charles Zheng

\section{Background}

\noindent\textbf{Theorem (Borell-TIS inequality)}
Let $f_t$ be a gaussian process such that $\mathbb{E}[f_t] = 0$
Then on any set $D$, and $u > 0$, 
\[
\mathbb{P}[\sup_D f_t > u + \mathbb{E}[\sup_D f_t]] \leq \exp(-u^2/(2\sigma_{max}^2))
\]
where
\[
\sigma_{max} = \sup_D \E[f_t^2]
\]

\noindent\textbf{Theorem (Slepian's inequality)}
If $f$ and $g$ are as bounded, centered gaussian processes, and
\[
\mathbb{E}[(f_t-f_s)^2] \leq \mathbb{E}[(g_t-g_s)^2]
\]
then
\[
\mathbb{P}[\sup_{t \in D} f_t  > u] \leq \mathbb{P}[\sup_{t \in D} g_t  > u]
\]

\noindent\textbf{Theorem.} Let $f$ be centered stationary Gaussian
process on a compact group $T$. Then the following three conditions
are equivalent: (i) $f_t$ is continuous (ii) $f_t$ is bounded (iii)
\[
\int_0^\infty \sqrt{H(\varepsilon)} d\varepsilon < \infty 
\]


\section{Supremum of an isotropic GP}

Let $f_t$ be a gaussian process on $\mathbb{R}^D$, with $\Cov(f_t,f_u)
= C(t-u)$, where $C(0) = 1$, and $C(t) \to 0$ as $||t|| \to \infty$.
Then if $f_t$ is bounded on an interval,
\[
\mathbb{P}\left(\lim_{T \to \infty}\frac{\sup_{[-T,T]^D}
    f_t}{\sqrt{2D\log(T)}} = 1
\right) = 1
\]

Sketch of proof:

\emph{1. Lower bound}.

Fix $T$. Let $\delta =exp(\sqrt{\log(T)})$ and consider
$t_1,...,t_{(2T/\delta)^D}$ on a square lattice of spacing $\delta$ on
$[-T,T]^D$.  Let $Z_{t_1},\hdots, Z_{t_{(2T/\delta)^D}}$ iid
$N(0,1-C(\delta))$.  By Slepian's inequality,
$\mathbb{P}(\max_{t_1,\hdots,t_{(2T/\delta)^D}} f_t > u) \geq
\mathbb{P}(\max_{t_1,\hdots,t_{(2T/\delta)^D}} Z_t > u)$.  Now note
that as $T \to \infty$, 
\[
\mathbb{P}\left(\lim_{T \to \infty}\frac{\max
    Z_t}{\sqrt{2D\log(T)}} = 1
\right) = 1
\]
This suggests, and with some more detailed analysis, implies that
\[
\mathbb{P}\left(\limsup_{T \to \infty}\frac{\sup_{[-T,T]^D}
    f_t}{\sqrt{2D\log(T)}} \geq 1
\right) = 1
\]

\emph{1. Upper bound}.

Partition $[-T,T]^D$ into hypercubes of edge length 1.
By union bound and Borrell-TIS inequality,
\[\mathbb{P}(\sup_{[-T,T]^D} f_t \geq u) \leq (2T)^D
\mathbb{P}(\sup_{[-T,T]^D} f_t \geq u) \leq (2T)^D e^{-u^2/2}\]
This can be used to show
\[
\mathbb{P}\left(\limsup_{T \to \infty}\frac{\sup_{[-T,T]^D}
    f_t}{\sqrt{2D\log(T)}} \leq 1
\right) = 1
\]

\section{One-dimensional isotropic gaussian processes}

Now that 
\[
C_\kappa(t) = [1-\kappa|t|]_+
\]
is a covariance kernel. This implies the following:

Suppose $C(0) = 1$, $-C''(t) > 0$ for $t > 0$ and $-\int_0^\infty t
C'(t) <\infty$.
Then $C(t)$ is a covariance kernel for a gaussian process.
This is because one can find a mixture density $\rho(k)$ such that
\[
C(t) = \int_0^\infty C_\kappa(t) \rho(\kappa) d\kappa
\]

\section{Nonexistence of unbounded isotropic GP}

We attempt to show that if $C(t) = 0$, $C(t) \to \infty$ as $t \to
\infty$, and $C(t) > 0$ for $t \neq 0$, that $f_t$ is continuous and
therefore bounded on an interval.
Otherwise,
\[
\int_0^\infty \sqrt{H(\varepsilon)} d\varepsilon \to \infty
\]

This in turn implies something like
\[
\lambda(\{t: C(t) > 1-\epsilon\}) \sim \exp(-\exp(1/t))
\]

We can try to derive a contradiction.
Let $\rho(\kappa)$ be the reflected Fourier transform of $C(t)$, so
\[
C(t) = \int_0^\infty \rho(\kappa) \cos(\kappa t)
\]
Since $C(0)=1$, $\rho(\kappa)$ is a probability distribution.
We know that $\rho$ has no first moment because $C(t)$ is
nondifferentiable at 0.

It must be that $-C'(t) \geq \text{const}/t^2$ as $t \to 0$.
Meanwhile
\[
-C'(t) = \int_0^\infty \rho(\kappa) \kappa \sin(\kappa t) d\kappa
\]

We have to show that the intergral on the right can't possible tend to
$\text{const}/t$ as $t \to 0$.

We know that $\kappa \sin(\kappa t)$ is positive from 0 to
$\frac{\pi}{t}$.
Hence we have
\[
\int_0^{\frac{\pi}{t}}\kappa \sin(\kappa t) \rho(\kappa) d\kappa \sim
\text{const} 1/t
\]
for $t$ small.
It remains to show that the rest of the integral
\[
\int_{\frac{\pi}{t}}^\infty \kappa \sin(\kappa t) \rho(\kappa) d\kappa \sim
\text{const} 1/t
\]
can't be too large. For any particular $t$ in $[0,\delta]$ it could be
large, but it may be possible to show that the average value of the
above interval is small for $t \in [\delta/2,\delta]$, say.
Then we would have to show that this suffices to establish that
$\int_0^\infty \sqrt{H(\varepsilon)} d\varepsilon \to \infty$.


\section{References}

Adler RF, Taylor J. \emph{Random Fields and Geometry}

\end{document}