\documentclass[11pt]{article} 

% packages with special commands
\usepackage{amssymb, amsmath}
\usepackage{epsfig}
\usepackage{array}
\usepackage{ifthen}
\usepackage{color}
\usepackage{fancyhdr}
\usepackage{graphicx}
%\usepackage{mathtools}
\definecolor{grey}{rgb}{0.5,0.5,0.5}

\begin{document}
\newcommand{\tr}{\text{tr}}
\newcommand{\E}{\textbf{E}}
\newcommand{\diag}{\text{diag}}
\newcommand{\argmax}{\text{argmax}}
\newcommand{\Cov}{\text{Cov}}
\pagestyle{fancy}

\title{A principled approach to decoding}

\author{Charles Zheng and Yuval Benjamini}

\maketitle

\begin{abstract}
In functional MRI (fMRI) studies, one presents a sequence of $T$
(possibly repeated) stimuli parameterized by features $x^{(1)},
\hdots, x^{(T)}$, where each $x^{(i)}$ is a $p$-dimensional vector.
The time-varying MRI image is processed to yield corresponding
response profiles $y^{(1)}, \hdots, y^{(T)}$, where each $y^{(i)}$ is
a vector of $V$ voxel-specific responses.  The goal of these studies
is to understand the relationship between $x^{(t)}$ and $y^{(t)}$: this goal can
be subdivided into the subgoal of learning an \emph{encoding model},
which predicts the response $y$ given the simulus, and the subgoal of
learning a \emph{decoding model}, which reconstructs the stimulus
given the response $y$.  One could interpret both models as
multivariate regression problems, with encoding fitting a model of the
form $Y = f(X) + \epsilon$ and decoding fitting a model of the form $X
= g(Y) + \varepsilon$.  However, the regression formulation is not the
only interpretation of the encoding/decoding problem.  Notably, Kay
\emph{et al} treat the encoding problem as a linear model, but pose
the decoding problem as one of \emph{identification}: that is, given
stimuli-response pairs $(x^{[i_1]}, y^{(1)}), \hdots, (x^{[i_j]},
y^{(j)})$ where the unobserved $x^{[i]}$ lie in a known set of stimuli
$S = \{x^{[1]},\hdots, x^{[|S|]}\}$, correctly recover the labels
$i_1,\hdots, i_j$ given only the responses $y^{(1)}, \hdots, y^{(j)}$.
Furthermore, Kay \emph{et al} quantify the quality of the decoding
model by the classification rate for the identification problem when
$S$ is selected randomly from a larger database of images
$\mathcal{S}$ (Kay 2008, Vu 2011).  This approach is more suited for
the goal of identifying \emph{natural images} from fMRI responses, and
has been adopted by numerous fMRI studies (Chen 2013). Such studies
usually use a combination of multivariate linear or nonlinear models
and feature selection to implement the decoding model.  However, such
studies have not explicitly motivated their decoding models based on
the criterion of maxmizing correct classification for random stimuli
subsets.  We proposed a principled approach to decoding, wherein we
formulate a decoding model which optimally maximizes the
identification perfomance of the model.  Our approach is based on a
theoretical analysis of the identification perfomance of a linear
model, resulting in an approximate measure of identification
performance which can be tractably optimized in training data.
\end{abstract}

\section{Introduction}

\section{References}

\begin{itemize}
\item Kay, KN., Naselaris, T., Prenger, R. J., and Gallant, J. L.
  ``Identifying natural images from human brain
  activity''. \emph{Nature} (2008)
\item Vu, V. Q., Ravikumar, P., Naselaris, T., Kay, K. N., and Yu, B.
  ``Encoding and decoding V1 fMRI responses to natural images with
  sparse nonparametric models'', \emph{The Annals of Applied
    Statistics}. (2011)
\item Chen, M., Han, J,. Hu, X., Jiang, Xi., Guo, L. and Liu, T.
  ``Survey of encoding and decoding of visual stimulus via fMRI: an
  image analysis perspective.'' \emph{Brain Imaging and
    Behavior}. (2014)
\item Schoenmakers, S., Barth, M., Heskes, T., van Gerven, M.,
  ``Linear reconstruction of percieved images from human brain
  activity'' \emph{NeuroImage} (2013)
\end{itemize}

\end{document}

